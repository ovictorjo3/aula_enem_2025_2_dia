\documentclass[11pt]{beamer}
\usetheme{CambridgeUS}
\usecolortheme{dolphin}

\usepackage[utf8]{inputenc}
\usepackage[brazil]{babel}
\usepackage[T1]{fontenc}
\usepackage{xcolor}
\usepackage{amsmath}
\usepackage{amsfonts}
\usepackage{amssymb}
\usepackage{graphicx}
\usepackage{setspace}
\usepackage{ragged2e}

\setbeamertemplate{caption}[numbered]
\setbeamertemplate{navigation symbols}{}

% Controle do gabarito %
%%%%%%%%%%%%%%%%%%%%%
\newif\ifgab
\gabtrue
\newcommand{\gab}[1]{%
  \ifgab
    \textcolor{red!80!black}{\textbf{#1}}%
  \else
    #1%
  \fi
}
%%%%%%%%%%%%%%%%%%%%%
\author[Luiz Claudio \& Joao Victor \& Letícia]{% 
	Luiz Claudio\inst{1} Joao Victor\inst{2} \and Letícia Almeida\inst{2}}
\title[AULÃO - MATEMÁTICA]{AULÃO MATEMÁTICA - ENEM 2025 - 2º Dia}
\institute[]{% 
	\textsuperscript{1}  Professor, CETi Gilberto Mestrinho (CETi) \\
	\textsuperscript{2} Pibidianos, Instituto Federal do Amazonas (IFAM)
}
\date{\today} 
\titlegraphic{\includegraphics[width=0.5\textwidth]{imagens/novo_logo.png}}
%\subject{}

% ---------------------------------------------------------

\begin{document}
\justifying
\onehalfspacing 

\begin{frame}
    \titlepage
\end{frame}

\section{Questões - Letícia}

\begin{frame}{ENEM 2016 - Continuação}
			
	\begin{columns}[T]
		\begin{column}{0.48\textwidth}
			
			\justifying
			Como o motorista conhece o percurso, sabe que existem, até a chegada a seu destino, cinco postos de abastecimento de combustível, localizados a 150 km,187 km, 450 km, 500 km e 570 km do ponto de partida. \\
					
			Qual a máxima distância, em quilômetro, que poderá percorrer até ser necessário reabastecer o veículo, de modo a não ficar sem combustível na estrada?
		\end{column}
				
		\begin{column}{0.48\textwidth}
			\begin{enumerate}[a]
				\item 570.
				\item 500.
				\item 450.
				\item 187.
				\item 150.
			\end{enumerate}
		\end{column}
	\end{columns}
\end{frame}
		
\section{Questões - Joao Victor}

\begin{frame}{Tópicos}
	\begin{alertblock}{}
		\vspace{1mm}
		\begin{itemize}
			\item Função polinomial do 1º grau; \\
			\item \uncover<2->{Função polinomial do 2º grau;} \\
			\item \uncover<3->{Geometria plana;} \\
			\item \uncover<4->{Estatística.}
		\end{itemize}
	\end{alertblock}
\end{frame}

\section{Função afim}

\begin{frame}{ENEM 2024 - Questão 136 (Caderno 5 - Amarelo)}
	%função afim
    Uma empresa produz mochilas escolares sob encomenda. Essa empresa tem um custo total de produção, composto por um custo fixo, que não depende do númerode mochilas, mais um custo variável, que é proporcional ao número de mochilas produzidas. O custo total cresce de forma linear, e a tabela apresenta esse custo para três quantidades de mochilas produzidas.
    
    \begin{center}
    	\includegraphics[scale=0.5]{imagens/enem_mochila.png}
    \end{center} 
\end{frame}

\begin{frame}{ENEM 2024 - Questão 136 (Caderno 5 - Amarelo)}
	
	O custo total, em real, para a produção de 80 mochilas será:
	
	\begin{columns}
		\begin{column}{0.6\textwidth}
			\centering
			\includegraphics[scale=0.55]{imagens/enem_mochila.png}
		\end{column}
		
		\begin{column}{0.2\textwidth}
			
			\begin{itemize}
				\item [a)] $2 400,00$
				\item [b)] $2 520,00$
				\item [c)] {$2 550,00$} %
				\item [d)] $2 700,00$
				\item [e)] $2 800,00$
			\end{itemize}
		\end{column}
	\end{columns}
	
	
\end{frame}

\begin{frame}{ENEM 2024 - Questão 152 (Caderno 5 - Amarelo)}
	%função afim
	As receitas anuais obtidas por uma indústria no períodode 2014 a 2021, em milhão de reais, foram registradas, por pontos, em um gráfico. Nele, também está representada a reta que descreve a tendência de evolução das receitas. Essa reta pode ser utilizada para estimar as receitas dos anos seguintes. 
	
	\begin{center}
		\includegraphics[scale=0.38]{imagens/enem2024-receitas.png}
	\end{center} 
\end{frame}

\begin{frame}{ENEM 2024 - Questão 152 (Caderno 5 - Amarelo)}
	
	A estimativa da receita, em milhão de reais, dessaindústria, para o ano de 2026, obtida a partir dessa reta de tendência, é:
	
	\begin{columns}
		\begin{column}{0.6\textwidth}
			\centering
			\includegraphics[scale=0.45]{imagens/enem2024-receitas.png}
		\end{column}
		
		\begin{column}{0.2\textwidth}
			
			\begin{itemize}
				\item [a)] $7$
				\item [b)] $8$
				\item [c)] $9$
				\item [d)] $10$
				\item [e)] {$11$} %
			\end{itemize}
		\end{column}
	\end{columns}
	
\end{frame}

\begin{frame}{ENEM 2024 - Questão 156 (Caderno 5 - Amarelo)}
	%função afim
	O uso de aplicativos de transporte tem sido uma alternativa à população que busca preços mais competitivos para se locomover, principalmente nas grandes cidades. As formas usadas para determinar o valor cobrado por cada viagem variam de um aplicativo para outro, mas, em geral, o valor V a ser pago, em real, varia em função de: 
	
	\begin{itemize}
		\item tarifa base F: valor fixo, em real, cobrado no inícioda viagem;
		\item tempo T: tempo, em minuto, de duração da viagem;
		\item distância D: distância percorrida, em quilômetro.
	\end{itemize}
	
	Um desses aplicativos cobra R\$ 2,00 de valor fixo,acrescido de R\$ 0,26 por minuto de viagem e de R\$ 1,40 por quilômetro rodado.
\end{frame}

\begin{frame}{ENEM 2024 - Questão 156 (Caderno 5 - Amarelo)}
	%função afim
	Nessas condições, a expressão que fornece o valor V a ser pago por uma viagem desse aplicativo é:
	
	\begin{itemize}
		\item [a)] 2,00F + 0,26T + 1,40D
		\item [b)] 2,00 + 0,26T + 1,40D   %
		\item [c)] 2,00 + 0,26T + D
		\item [d)] 0,26T + 1,40D
		\item [e)] F + T + D
	\end{itemize}
	
\end{frame}

\section{Função Quadrática}

\begin{frame}{ENEM 2013}
	
	A parte interior de uma taça foi gerada pela rotação de uma parábola em torno de um eixo z, conforme mostra a figura. A função real que expressa a parábola, no plano cartesiano da figura, é dada pela lei: $f(x)=\dfrac{3}{2}x^{2}-6x+c$ onde C é a medida da altura do líquido contido na taça, em centímetros. Sabe-se que o ponto V, na figura, representa o vértice da parábola, localizado sobre o eixo x. 
\end{frame}

\begin{frame}{ENEM 2013 (continuação)}
	Nessas condições, a altura do líquido contido na taça, em centímetros, é:
	\begin{columns}
		\begin{column}{0.6\textwidth}
			\centering
			\includegraphics[width=0.7\linewidth]{imagens/enem 2013.png}
			
		\end{column}
	
		\begin{column}{0.4\textwidth}
			\begin{itemize}
				\item [a)] $1$ 
				\item [b)] $2$  
				\item [c)] $4$
				\item [d)] $5$ 
				\item [e)] {$6$} %
			\end{itemize}
		\end{column}
	\end{columns}
\end{frame}

\begin{frame}{ENEM 2014}
	
	Um professor, depois de corrigir as provas de sua turma, percebeu que várias questões estavam muito difíceis. Para compensar, decidiu utilizar uma função polinomial f, de grau menor que 3, para alterar as notas x da prova para notas $y = f(x)$, da seguinte maneira:
	
	\begin{enumerate}[I]
		\item A nota zero permanece zero.
		\item A nota 10 permanece 10.
		\item A nota 5 passa a ser 6.
	\end{enumerate}
	
\end{frame}

\begin{frame}{ENEM 2014 (continuação)}
	A expressão da função $y = f(x)$ a ser utilizada pelo professor é:
	
	\begin{itemize}
		\item [a)] {$y=-\dfrac{1}{25}x^{2}+\dfrac{7}{5}x$} % \\
		\item [b)] $y=-\dfrac{1}{10}x^{2}+2x$ \\
		\item [c)] $y=\dfrac{1}{24}x^{2}+\dfrac{7}{12}x$ \\
		\item [d)] $y=\dfrac{4}{5}x+2$ \\
		\item [e)] $y=x$ 
	\end{itemize}
	
\end{frame}

\section{Estatistica}

\begin{frame}{ENEM 2024 - Questão 164 (Caderno 5 - Amarelo)}
	%media 
	Ao calcular a média de suas notas em 4 provas, um estudante dividiu, por engano, a soma das notas por 5. Com isso, a média obtida foi 1 unidade menor do que deveria ser, caso fosse calculada corretamente. O valor correto da média das notas desse estudante é:
	
	\begin{itemize}
		\item [a)] 4
		\item [b)] 5   %
		\item [c)] 6
		\item [d)] 19
		\item [e)] 21
	\end{itemize}
	
\end{frame}

\begin{frame}{ENEM 2024 - Questão 137 (Caderno 5 - Amarelo)}
	%mediana
	A umidade relativa do ar é um dos indicadores utilizados na meteorologia para fazer previsões sobre o clima. O quadro apresenta as médias mensais, em porcentagem, da umidade relativa do ar em um período de seis meses
	consecutivos em uma cidade.
	
	\begin{center}
		\includegraphics[scale=0.5]{imagens/enem2024-umidade.png}
	\end{center} 
\end{frame}

\begin{frame}{ENEM 2024 - Questão 137 (Caderno 5 - Amarelo)}

	Nessa cidade, a mediana desses dados, em porcentagem,
	da umidade relativa do ar no período considerado foi:
	
	\begin{columns}
		\begin{column}{0.6\textwidth}
			\centering
			\includegraphics[scale=0.5]{imagens/enem2024-umidade.png}
		\end{column}
		
		\begin{column}{0.2\textwidth}
			
			\begin{itemize}
				\item [a)] $56$
				\item [b)] $58$
				\item [c)] $59$
				\item [d)] $60$
				\item [e)] {$62$} %
			\end{itemize}
		\end{column}
	\end{columns}
\end{frame}

\begin{frame}{ENEM 2024 - Questão 154 (Caderno 5 - Amarelo)}
	%media aritmetica
	Contratos de vários serviços disponíveis na internet apresentam uma quantidade excessiva de informações. Isso faz com que o tempo necessário para a leitura desses contratos possa ser longo. O quadro apresenta uma amostra do tempo considerado necessário para a leitura completa do contrato de alguns
	serviços digitais.
	
	\begin{center}
		\includegraphics[scale=0.5]{imagens/enem2024-contratos.png}
	\end{center} 
\end{frame}

\begin{frame}{ENEM 2024 - Questão 154 (Caderno 5 - Amarelo)}
	
	O tempo médio, em minuto, necessário para a leitura completa de um contrato de serviço dentre os listados no quadro é, com uma casa decimal, aproximadamente,
	
	\begin{columns}
		\begin{column}{0.6\textwidth}
			\centering
			\includegraphics[scale=0.5]{imagens/enem2024-contratos.png}
		\end{column}
		
		\begin{column}{0.2\textwidth}
			
			\begin{itemize}
				\item [a)] $13,0$
				\item [b)] $15,0$
				\item [c)] {$19,8$} %
				\item [d)] $20,0$ 
				\item [e)] $23,3$ 
			\end{itemize}
		\end{column}
	\end{columns}
\end{frame}


\section{Geometria Plana}

\begin{frame}{ENEM 2024 - Questão 172 (Caderno 5 - Amarelo)}
	%Área de quadriláteros
	O estádio do Maracanã passou por algumas modificações estruturais para realização da Copa doMundo de 2014, como, por exemplo, as dimensões do campo retangular. Para se adaptar aos padrões da Fifa, as dimensões do campo foram reduzidas de 110 m × 75 m para 105 m × 68 m. Em quantos metros quadrados a área do campo doMaracanã foi reduzida?
	
	\begin{itemize}
		\item [a)] 24
		\item [b)] 35
		\item [c)] 555
		\item [d)] 1110 %
		\item [e)] 1145
	\end{itemize}
	
\end{frame}

\begin{frame}{ENEM 2013}
	O dono de um sítio pretende colocar uma haste de sustentação para melhor firmar dois postes de comprimentos iguais a 6m e 4m. A figura representa a situação real na qual os postes são descritos pelos segmentos AC e BD e a haste é representada pelo EF, todos perpendiculares ao solo, que é indicado pelo segmento de reta AB. Os segmentos AD e BC representam cabos de aço que
	serão instalados. 
	
	\begin{center}
		\includegraphics[scale=0.39]{imagens/enem_2013_sitio.png}
	\end{center} 
\end{frame}

\begin{frame}{ENEM 2013 (continuação)}
	Qual deve ser o valor do comprimento da haste EF?
	
	\begin{columns}
		\begin{column}{0.6\textwidth}
			\centering
			\includegraphics[scale=0.35]{imagens/enem_2013_sitio.png}
		\end{column}
		
		\begin{column}{0.2\textwidth}
			
			\begin{itemize}
				\item [a)] 1 m
				\item [b)] 2 m
				\item [c)] 2,4 m %
				\item [d)] 3 m 
				\item [e)] $2\sqrt{6}$ m
			\end{itemize}
		\end{column}
	\end{columns}

\end{frame}

\end{document}

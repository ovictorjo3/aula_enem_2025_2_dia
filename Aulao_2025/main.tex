\documentclass[11pt]{beamer}
\usetheme{CambridgeUS}
\usecolortheme{dolphin}

\usepackage[utf8]{inputenc}
\usepackage[brazil]{babel}
\usepackage[T1]{fontenc}
\usepackage{xcolor}
\usepackage{amsmath}
\usepackage{amsfonts}
\usepackage{amssymb}
\usepackage{graphicx}
\usepackage{setspace}
\usepackage{ragged2e}

\setbeamertemplate{caption}[numbered]

% Controle do gabarito %
%%%%%%%%%%%%%%%%%%%%%
\newif\ifgab
\gabtrue
\newcommand{\gab}[1]{%
  \ifgab
    \textcolor{red!80!black}{\textbf{#1}}%
  \else
    #1%
  \fi
}
%%%%%%%%%%%%%%%%%%%%%
\author[Luiz Claudio \& Joao Victor \& Letícia]{% 
	Luiz Claudio\inst{1} Joao Victor\inst{2} \and Letícia\inst{2}}
\title[AULÃO - MATEMÁTICA]{AULÃO MATEMÁTICA- ENEM 2025 - 2º Dia}
\institute[]{% 
	\textsuperscript{1}  Professor, CETi Gilberto Mestrinho (CETi) \\
	\textsuperscript{2} Pibidianos, Instituto Federal do Amazonas (IFAM)
}
\date{\today} 
\titlegraphic{\includegraphics[width=0.5\textwidth]{imagens/novo_logo.png}}
%\subject{}

% ---------------------------------------------------------

\begin{document}
\justifying
\onehalfspacing 

\begin{frame}
    \titlepage
\end{frame}

\section{Questões - Letícia}

\begin{frame}{ENEM 2013}
	(Enem 2013) Em um certo teatro, as poltronas são divididas em setores. A figura apresenta a vista do setor 3 desse teatro, no qual as cadeiras escuras estão reservadas e as claras não foram vendidas.

	\centering
    \includegraphics[width=0.7\textwidth]{arquivos_snct/cadeiras.png}

\ens{frame}

\begin{frame}{ENEM 2013 - Continuação}

	A razão que representa a quantidade de cadeiras reservadas do setor 3 em relação ao total de cadeiras desse mesmo setor é

	\begin{enumerate}[a]
		\item $\frac{17){70}$
		\item $\frac{17}{53}$
		\item $\frac{53}{70}$
		\item $\frac{53}{17}$
		\item $\frac{70}{17}$
	\end{enumerate}
\end{frame}
	
\begin{frame}{ENEM 2010}

	Uma empresa vende tanques de combustíveis de formato cilíndrico, em três tamanhos, com medidas indicadas nas figuras. O preço do tanque é diretamente proporcional à medida da área da superfície lateral do tanque. O dono de um posto de combustível deseja encomendar um tanque com menor custo por metro cúbico de capacidade de armazenamento.
	
	\centering
	    \includegraphics[width=0.7\textwidth]{arquivos_snct/tanques.png}
	
\end{frame}

\begin{frame}{ENEM 2010 - Continuação}
    
	Qual dos tanques deverá ser escolhido pelo dono do posto? (Considere $\pi = 3$)

    \begin{enumerate}[a]
        \item I, pela relação área/capacidade de armazenamento de $\frac{1}{3}$.
        \item I, pela relação área/capacidade de armazenamento de $\frac{4}{3}$.
        \item II, pela relação área/capacidade de armazenamento de $\frac{3}{4}$.
        \item III, pela relação área/capacidade de armazenamento de $\frac{2}{3}$.
        \item III, pela relação área/capacidade de armazenamento de $\frac{7}{12}$.
    \end{description}
\end{frame}

\begin{frame}{ENEM 2010}
     
	 A loja Telas & Molduras cobra 20 reais por metro quadrado de tela, 15 reais por metro linear de moldura, mais uma taxa fixa de entrega de 10 reais.
	 Uma artista plástica precisa encomendar telas e molduras a essa loja, suficientes para 8 quadros retangulares (25 cm x 50 cm). Em seguida, fez uma segunda encomenda, mas agora para 8 quadros retangulares (50 cm x 100 cm). 
\end{frame}

\begin{frame}{ENEM 2010 - Continuação}
          
		  O valor da segunda encomenda será
     \begin{enumerate}[a]
        \item o dobro do valor da primeira encomenda, porque a altura e a largura dos quadros dobraram.
        \item maior do que o valor da primeira encomenda, mas não o dobro.
        \item a metade do valor da primeira encomenda, porque a altura e a largura dos quadros dobraram.
        \item menor do que o valor da primeira encomenda, mas não a metade.
        \item igual ao valor da primeira encomenda, porque o custo de entrega será o mesmo. 
    \end{description}
     \end{frame}

\begin{frame}{ENEM 2011	}

	\begin{columns}[T]
    \begin{column}{0.48\textwidth}
	Para uma atividade realizada no laboratório de Matemática, um aluno precisa construir uma maquete da quadra de esportes da escola que tem 28 m de comprimento por 12 m de largura. A maquete deverá ser construída na escala de 1 : 250. Que medidas de comprimento e largura, em cm, o aluno utilizará na construção da maquete? 
 \end{column}
    \begin{column}{0.48\textwidth}
     \begin{enumerate}[a]
        \item $4,8$ e $11,2$.
        \item $7,0$ e $3,0$.
        \item $11,2$ e $4,8$.
        \item $28,0$ e $12,0$.
        \item $30,0$ e $70,0$. 
    \end{description}
    \end{column}
  \end{columns}
\end{frame}

\begin{frame}{ENEM 2010}

\begin{columns}[T] 
    \begin{column}{0.48\textwidth}
 Em 2006, a produção mundial de etanol foi de 40 bilhões de litros e a de biodiesel, de 6,5 bilhões. Neste mesmo ano, a produção brasileira de etanol correspondeu a 43\% da produção mundial, ao passo que a produção dos Estados Unidos da América, usando milho, foi de 45\%.

 Considerando que, em 2009, a produção mundial de etanol seja a mesma de 2006 e que os Estados Unidos produzirão somente a metade de sua produção de 2006, para que o total produzido pelo  
    \end{column}

     \begin{column}{0.48\textwidth}
     Brasil e pelos Estados Unidos continue correspondendo a 88\% da produção mundial, o Brasil deve aumentar sua produção em, aproximadamente,
        \begin{enumerate}[a]
            \item 22,5\%.
            \item 50,0\%.
            \item 52,3\%.
            \item 65,5\%.
            \item 77,5\%
        \end{enumerate}
        \end{column}
        \end{columns}
\end{frame}

    \begin{frame}{ENEM 2011}
    \begin{columns}[T] 
    \begin{column}{0.48\textwidth}
        Uma pessoa aplicou certa quantia em ações. No primeiro mês, ela perdeu 30\% do total do investimento e, no segundo mês, recuperou 20\% do que havia perdido.

Depois desses dois meses, resolveu tirar o montante de R\$ 3800,00 gerado pela aplicação. 
\end{column}

     \begin{column}{0.48\textwidth}
A quantia inicial que essa pessoa aplicou em ações corresponde ao valor de 
    \begin{enumerate}[a]
        \item R\$ 4.222,22.
        \item R\$ 4.523,80.
        \item R\$ 5.000,00.
        \item R\$ 13.300,00.
        \item R\$ 17.100,00.
    \end{enumerate}
 \end{column}
    \end{columns}
    \end{frame}

    \begin{frame}{ENEM 2016}
        (Enem 2016) Uma pessoa comercializa picolés. No segundo dia de certo evento ela comprou 4 caixas de picolés, pagando R\$ 16,00 a caixa com 20 picolés para revendê-los no evento. No dia anterior, ela havia comprado a mesma quantidade de picolés, pagando a mesma quantia, e obtendo um lucro de R\$ 40,00 (obtido exclusivamente pela diferença entre o valor de venda e o de compra dos picolés) com a venda de todos os picolés que possuía.

        Pesquisando o perfil do público que estará presente no evento, a pessoa avalia que será possível obter um lucro 20\% maior do que o obtido com a venda no primeiro dia do evento.
    \end{frame}

    \begin{frame}{ENEM 2016 - Continuação}
        Para atingir seu objetivo, e supondo que todos os picolés disponíveis foram vendidos no segundo dia, o valor de venda de cada picolé, no segundo dia, deve ser

        \begin{enumerate}[a]
            \item R\$ 0,96.
            \item R\$ 1,00.
            \item R\$ 1,40.
            \item R\$ 1,50.
            \item R\$ 1,56.
        \end{enumerate}
    \end{frame}

    \begin{frame}{ENEM 2014}

        \begin{columns}[T] 
    \begin{column}{0.48\textwidth}
Um show especial de Natal teve 45.000 ingressos vendidos. Esse evento ocorrerá em um estádio de futebol que disponibilizará 5 portões de entrada, com 4 catracas eletrônicas por portão. Em cada uma dessas catracas, passará uma única pessoa a cada 2 segundos. O público foi igualmente dividido pela quantidade de portões e catracas, indicados no ingresso para o show, para a efetiva entrada no estádio. Suponha que todos aqueles que compraram ingressos irão ao show e que todos passarão pelos

    \end{column}

     \begin{column}{0.48\textwidth}
     portões e catracas eletrônicas indicados.
     
     Qual é o tempo mínimo para que todos passem pelas catracas?

     \begin{enumerate}[a]
         \item 1 hora.
         \item 1 hora e 15 minutos.
         \item 5 horas.
         \item 6 horas.
         \item 6 horas e 15 minutos.
     \end{enumerate}
        \end{column}
    \end{columns}
    \end{frame}

    \begin{frame}{ENEM 2015}
        A insulina é utilizada no tratamento de pacientes com diabetes para o controle glicêmico. Para facilitar sua aplicação, foi desenvolvida uma “caneta” na qual pode ser inserido um refil contendo 3mL de insulina, como mostra a imagem.

            \centering
            \includegraphics[width=0.7\linewidth]{insulina.png}

    \end{frame}
\begin{frame}{ENEM 2015 - Continuação}

 \begin{columns}[T] 
    \begin{column}{0.48\textwidth}
Para controle das aplicações, definiu-se a unidade de insulina como 0,01mL. Antes de cada aplicação, é necessário descartar 2 unidades de insulina, de forma a retirar possíveis bolhas de ar. 

A um paciente foram prescritas duas aplicações diárias: 10 unidades de insulina pela manhã e 10 à noite.
 \end{column}

     \begin{column}{0.48\textwidth}

Qual o número máximo de aplicações por refil que o paciente poderá utilizar com a dosagem prescrita? 

\begin{enumerate}[a]
    \item 25.
    \item 15.
    \item 13.
    \item 12.
    \item 8. 
\end{enumerate}
\end{column}
    \end{columns}
\end{frame}

\begin{frame}{ENEM 2016}

    No tanque de um certo carro de passeio cabem até 50 L de combustível, e o rendimento médio deste carro na estrada é de 15 km L de combustível. Ao sair para uma viagem de 600 km o motorista observou que o marcador de combustível estava exatamente sobre uma das marcas da escala divisória do medidor, conforme figura a seguir.

     \centering
            \includegraphics[width=0.7\linewidth]{carrinho.png}
\end{frame}

\begin{frame}{ENEM 2016 - Continuação}

    \begin{columns}[T]
        \begin{column}{0.48\textwidth}
        Como o motorista conhece o percurso, sabe que existem, até a chegada a seu destino, cinco postos de abastecimento de combustível, localizados a 150 km,187 km, 450 km, 500 km e 570 km do ponto de partida.

Qual a máxima distância, em quilômetro, que poderá percorrer até ser necessário reabastecer o veículo, de modo a não ficar sem combustível na estrada?
    \end{column}
     \begin{column}{0.48\textwidth}
     \begin{enumerate}[a]
        \item 570.
        \item 500.
        \item 450.
        \item 187.
        \item 150.
     \end{enumerate}
     \end{column}
    \end{columns}
\end{frame}

\section{Questões - Joao Victor}

\begin{frame}{ENEM 2013}
    
    A parte interior de uma taça foi gerada pela rotação de uma parábola em torno de um eixo z, conforme mostra a figura. A função real que expressa a parábola, no plano cartesiano da figura, é dada pela lei: $f(x)=\dfrac{3}{2}x^{2}-6x+c$ onde C é a medida da altura do líquido contido na taça, em centímetros. Sabe-se que o ponto V, na figura, representa o vértice da parábola, localizado sobre o eixo x. 

    \vfill
    \textbf{(continua no próximo slide)}
    
\end{frame}

\begin{frame}{ENEM 2013 (continuação)}
    Nessas condições, a altura do líquido contido na taça, em centímetros, é:
    \begin{columns}
        \begin{column}{0.4\textwidth}
            \begin{enumerate}[a]
                \item $1$ 
                \item $2$  
                \item $4$
                \item $5$ 
                \item \gab{$6$} %
            \end{enumerate}
        \end{column}

        \begin{column}{0.6\textwidth}
            \centering
            \includegraphics[width=0.7\linewidth]{imagens/enem 2013.png}
        \end{column}
    \end{columns}
    
\end{frame}

\begin{frame}{ENEM 2014}

    Um professor, depois de corrigir as provas de sua turma, percebeu que várias questões estavam muito difíceis. Para compensar, decidiu utilizar uma função polinomial f, de grau menor que 3, para alterar as notas x da prova para notas $y = f(x)$, da seguinte maneira:

    \begin{enumerate}[I]
        \item A nota zero permanece zero.
        \item A nota 10 permanece 10.
        \item A nota 5 passa a ser 6.
    \end{enumerate}

    \vfill
    \textbf{(continua no próximo slide)}
    
\end{frame}

\begin{frame}{ENEM 2014 (continuação)}
    A expressão da função $y = f(x)$ a ser utilizada pelo professor é:

    \begin{enumerate}[a]
        \item \gab{$y=-\dfrac{1}{25}x^{2}+\dfrac{7}{5}x$} % \\
        \item $y=-\dfrac{1}{10}x^{2}+2x$ \\
        \item $y=\dfrac{1}{24}x^{2}+\dfrac{7}{12}x$ \\
        \item $y=\dfrac{4}{5}x+2$ \\
        \item $y=x$ 
    \end{enumerate}
    
\end{frame}

\begin{frame}{ENEM PPL 2019}
    No desenvolvimento de um novo remédio, pesquisadores monitoram a quantidade Q de uma substância circulando na corrente sanguínea de um paciente, ao longo do tempo t. Esses pesquisadores controlam o processo, observando que Q é uma função quadrática de t. Os dados coletados nas duas primeiras horas foram:

    \begin{center}
        \includegraphics[scale=0.5]{imagens/enem-ppl-2019.png}
    \end{center} \textbf{(continua no próximo slide)}
\end{frame}

\begin{frame}{ENEM PPL 2019 (continuação)}
    Para decidir se devem interromper o processo, evitando riscos ao paciente, os pesquisadores querem saber, antecipadamente, a quantidade da substância que estará circulando na corrente sanguínea desse paciente após uma hora do último dado coletado. Nas condições expostas, essa quantidade (em miligrama) será igual a:

    \begin{enumerate}[a]
        \item $4$
        \item \gab{$7$} %
        \item $8$ 
        \item $9$ 
        \item $10$ 
    \end{enumerate}
\end{frame}

\begin{frame}{ENEM 2022}

    Ao analisar os dados de uma epidemia em uma cidade, peritos obtiveram um modelo que avalia a quantidade de pessoas infectadas a cada mês, ao longo de um ano. O modelo é dado por $p(t)=-t^{2}+10t+24$, sendo t um número natural, variando de 1 a 12, que representa os meses do ano, e p(t) a quantidade de pessoas infectadas no mês t do ano. Para tentar diminuir o número de infectados no próximo ano, a Secretaria Municipal de Saúde decidiu intensificar a propaganda oficial sobre os cuidados com a epidemia. Foram apresentadas cinco propostas (I, II, III, IV e V), com diferentes períodos de intensificação das propagandas: $\textbf{(I)}\ 1 \leq t \leq 2, \textbf{(II)}\ 3 \leq t \leq 4, \textbf{(III)}\ 5 \leq t \leq 6, \textbf{(IV)}\ 7 \leq t \leq 9$ e $\textbf{(V)}\ 10 \leq t \leq 12$

    \vfill
    \textbf{(continua no próximo slide)}
\end{frame}
\begin{frame}{ENEM 2022 (continuação)}
    A sugestão dos peritos é que seja escolhida a proposta cujo período de intensificação da propaganda englobe o mês em que, segundo o modelo, há a maior quantidade de infectados. A sugestão foi aceita. A proposta escolhida foi a:

    \begin{enumerate}[a]
            \item I
            \item II
            \item \gab{III} %
            \item IV 
            \item V
        \end{enumerate}
\end{frame}

\begin{frame}{UEA - 2014}

    Em um sistema de coordenadas cartesianas ortogonais, os gráficos das funções quadráticas $f(x) = -2x^{2} + 5x + 3$ e $g(x)$ são simétricos um do outro com relação ao eixo das abscissas. Desse modo, é correto afirmar que a distância entre os seus vértices é igual a:

    \begin{enumerate}[a]
        \item 12,05
        \item 6,25
        \item 8,15
        \item 6,12
        \item \gab{12,25} %
    \end{enumerate}
    
\end{frame}

\begin{frame}{UEA - 2015}
    Considere o gráfico da função quadrática dada por $f(x) = -x^{2} + 2x + c$. De acordo com o gráfico, é correto afirmar que o valor de y é:
    \begin{columns}
        \begin{column}{0.4\textwidth}
            \begin{enumerate}[a]
                \item negativo, se $x < 3$
                \item positivo, se $x > 3$
                \item \gab{positivo, no intervalo $ -1 < x < 3$} %
                \item negativo, se $x > -1$
                \item zero, se $x = 1$
            \end{enumerate}
        \end{column}

        \begin{column}{0.6\textwidth}
            \centering
            \includegraphics[width=0.7\linewidth]{imagens/uea-2015.png}
        \end{column}
    \end{columns}
    
\end{frame}

\begin{frame}{UEA - 2017}
    O gráfico da função real $f(x)=ax^{2}+ bx + c$, com $a > 0$, é a parábola representada na figura. Sabendo-se que $x_{1}+x_{2}= -{b}/2$, onde $ x_{1}, x_ {2}$ são as raízes de $f(x) = 0$, é correto afirmar que a parábola intersecta o eixo das ordenadas no ponto:

    \begin{columns}
        \begin{column}{0.4\textwidth}
            \begin{enumerate}[a]
                \item $(0,12)$ 
                \item $(12,0)$
                \item $(0,4)$ 
                \item \gab{$(0,16)$} %
                \item $(16,0)$
            \end{enumerate}
        \end{column}

        \begin{column}{0.5\textwidth}
            \centering
            \includegraphics[width=0.8\linewidth]{imagens/uea-macro-2017(2).png}
        \end{column}
    \end{columns}
    
\end{frame}

\begin{frame}{Vunesp}
    Uma função quadrática tem o eixo dos y como eixo de simetria. A distância entre os zeros da função é de 4 unidades, e a função tem $-5$ como valor mínimo. Esta função quadrática é:

    \begin{enumerate}[a]
        \item $y=5x^{2}-4x-5$
        \item $y=5x^{2}-20$ \\
        \item $y=\dfrac{5}{4}x^{2}-5x$ \\
        \item \gab{$y=\dfrac{5}{4}x^{2}-5$} \\ %
        \item $y=\dfrac{5}{4}x^{2}-20$
    \end{enumerate}
\end{frame}

\begin{frame}{IFPE 2019}
    Um balão de ar quente sai do solo às 9h da manhã (origem do sistema cartesiano) e retorna ao solo 8 horas após sua saída, conforme demons- trado a seguir. A altura h, em metros, do balão, está em função do tempo t, em horas, através da fórmula $h(t)=(-{3}/{4})t^{2}+6t$ .

    \begin{center}
        \includegraphics[scale=0.5]{imagens/IFPE 2019.png}
    \end{center}
\end{frame}

\begin{frame}{IFPE 2019 (continuação)}
    A altura máxima atingida pelo balão é de:

    \begin{enumerate}[a]
        \item 21 m 
        \item 36 m
        \item 8 m 
        \item 4 m
        \item \gab{12 m} %
    \end{enumerate}
\end{frame}

\begin{frame}{AFA}
    Para que o valor mínimo da função $y=x^{2}-4x+k$ seja igual a $-1$, o valor de k é:

    \begin{enumerate}[a]
        \item 1
        \item 2
        \item \gab{3} %
        \item 4
        \item 5
    \end{enumerate}
\end{frame}

\begin{frame}{PUC - Campinas SP - 2015}
    A figura indica um bombeiro lançando um jato de água para apagar o fogo em um ponto de uma torre retilínea e perpendicular ao chão. A trajetória do jato de água é parabólica, e dada pela função $y = -x^{2} + 2x + 3$ , com x e y em metros. Sabendo que o ponto de fogo atingido pelo jato de água está a 2 metros do chão, então, qual o valor de $p-q$, em metros?

    \begin{columns}
        \begin{column}{0.4\textwidth}
            \begin{enumerate}[a]
                \item $2+\sqrt{2}$ 
                \item $1+\sqrt{2}$  
                \item $4-2\sqrt{2}$
                \item $3-\sqrt{2}$ 
                \item \gab{$2-\sqrt{2}$} %
            \end{enumerate}
        \end{column}

        \begin{column}{0.6\textwidth}
            \centering
            \includegraphics[width=0.75\linewidth]{imagens/q20.png}
        \end{column}
    \end{columns}
\end{frame}

\begin{frame}{Unifor-CE}
    Na figura abaixo têm-se os gráficos das funções quadráticas f e g. Se P é um dos pontos de interseção de f e g, então as suas coordenadas são:

    \begin{columns}
        \begin{column}{0.4\textwidth}
            \begin{enumerate}[a]
                \item $(-{3}/{4},{57}/{16})$ 
                \item \gab{$(-{1}/{2},{9}/{4})$}  %
                \item $(-{1}/{2},-{9}/{4})$
                \item $(-{1}/{4},{17}/{16})$ 
                \item $(-{1}/{4},-{17}/{16})$
            \end{enumerate}
        \end{column}

        \begin{column}{0.6\textwidth}
            \centering
            \includegraphics[width=0.9\linewidth]{imagens/Unifor-CE.png}
        \end{column}
    \end{columns}
    
\end{frame}

\end{document}

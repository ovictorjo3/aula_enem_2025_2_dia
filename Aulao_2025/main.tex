\documentclass[11pt]{beamer}
\usetheme{CambridgeUS}
\usecolortheme{dolphin}

\usepackage[utf8]{inputenc}
\usepackage[brazil]{babel}
\usepackage[T1]{fontenc}
\usepackage{xcolor}
\usepackage{amsmath}
\usepackage{amsfonts}
\usepackage{amssymb}
\usepackage{graphicx}
\usepackage{setspace}
\usepackage{ragged2e}

\setbeamertemplate{caption}[numbered]
\setbeamertemplate{navigation symbols}{}

% Controle do gabarito %
%%%%%%%%%%%%%%%%%%%%%
\newif\ifgab
\gabtrue
\newcommand{\gab}[1]{%
  \ifgab
    \textcolor{red!80!black}{\textbf{#1}}%
  \else
    #1%
  \fi
}
%%%%%%%%%%%%%%%%%%%%%
\author[Luiz Claudio \& Joao Victor \& Letícia]{% 
	Luiz Claudio\inst{1} Joao Victor\inst{2} \and Letícia Almeida\inst{2}}
\title[AULÃO - MATEMÁTICA]{AULÃO MATEMÁTICA - ENEM 2025 - 2º Dia}
\institute[]{% 
	\textsuperscript{1}  Professor, CETi Gilberto Mestrinho (CETi) \\
	\textsuperscript{2} Pibidianos, Instituto Federal do Amazonas (IFAM)
}
\date{\today} 
\titlegraphic{\includegraphics[width=0.5\textwidth]{imagens/novo_logo.png}}
%\subject{}

% ---------------------------------------------------------

\begin{document}
\justifying
\onehalfspacing 

\begin{frame}
    \titlepage
\end{frame}

\section{Questões - Letícia}


\section{Questões - Joao Victor}

\begin{frame}{Tópicos}
	\begin{alertblock}{}
		\vspace{1mm}
		\begin{itemize}
			\item Função polinomial do 1º grau; \\
			\item \uncover<2->{Função polinomial do 2º grau;} \\
			\item \uncover<3->{Geometria plana;} \\
			\item \uncover<4->{Estatística.}
		\end{itemize}
	\end{alertblock}
\end{frame}

\begin{frame}{ENEM 2024 - Questão 136 (Caderno 5 - Amarelo)}
	%função afim
    Uma empresa produz mochilas escolares sob encomenda. Essa empresa tem um custo total de produção, composto por um custo fixo, que não depende do númerode mochilas, mais um custo variável, que é proporcional ao número de mochilas produzidas. O custo total cresce de forma linear, e a tabela apresenta esse custo para três quantidades de mochilas produzidas.
    
    \begin{center}
    	\includegraphics[scale=0.5]{imagens/enem_mochila.png}
    \end{center} O custo total, em real, para a produção de 80 mochilas será:
\end{frame}

\begin{frame}{ENEM 2024 - Questão 136 (Caderno 5 - Amarelo)}
	
	\begin{center}
		\includegraphics[scale=0.5]{imagens/enem_mochila.png}
	\end{center} 
	
	\begin{itemize}
		\item [a)] $2 400,00$
		\item [b)] $2 520,00$
		\item [c)] \gab{$2 550,00$} %
		\item [d)] $2 700,00$
		\item [e)] $2 800,00$
	\end{itemize}
	
\end{frame}


\begin{frame}{ENEM 2024 - Questão 137 (Caderno 5 - Amarelo)}
	%mediana
	A umidade relativa do ar é um dos indicadores utilizados na meteorologia para fazer previsões sobre o clima. O quadro apresenta as médias mensais, em porcentagem, da umidade relativa do ar em um período de seis meses
	consecutivos em uma cidade.
	
	\begin{center}
		\includegraphics[scale=0.5]{imagens/enem2024-umidade.png}
	\end{center} Nessa cidade, a mediana desses dados, em porcentagem,
	da umidade relativa do ar no período considerado foi:
\end{frame}

\begin{frame}{ENEM 2024 - Questão 137 (Caderno 5 - Amarelo)}

	\begin{center}
		\includegraphics[scale=0.5]{imagens/enem2024-umidade.png}
	\end{center} 
	
	\begin{itemize}
		\item [a)] $56$
		\item [b)] $58$
		\item [c)] $59$
		\item [d)] $60$
		\item [e)] \gab{$62$} %
	\end{itemize}
\end{frame}

\begin{frame}{ENEM 2024 - Questão 152 (Caderno 5 - Amarelo)}
	%função afim
	As receitas anuais obtidas por uma indústria no períodode 2014 a 2021, em milhão de reais, foram registradas, por pontos, em um gráfico. Nele, também está representada a reta que descreve a tendência de evolução das receitas.Essa reta pode ser utilizada para estimar as receitas dos anos seguintes. A estimativa da receita, em milhão de reais, dessaindústria, para o ano de 2026, obtida a partir dessa reta de tendência, é
	
	\begin{center}
		\includegraphics[scale=0.28]{imagens/enem2024-receitas.png}
	\end{center} 
\end{frame}

\begin{frame}{ENEM 2024 - Questão 152 (Caderno 5 - Amarelo)}
	\begin{center}
		\includegraphics[scale=0.33]{imagens/enem2024-receitas.png}
	\end{center}
	
	\begin{itemize}
		\item [a)] $7$
		\item [b)] $8$
		\item [c)] $9$
		\item [d)] $10$
		\item [e)] \gab{$11$} %
	\end{itemize}
	
	
\end{frame}


\begin{frame}{ENEM 2013}
    
    A parte interior de uma taça foi gerada pela rotação de uma parábola em torno de um eixo z, conforme mostra a figura. A função real que expressa a parábola, no plano cartesiano da figura, é dada pela lei: $f(x)=\dfrac{3}{2}x^{2}-6x+c$ onde C é a medida da altura do líquido contido na taça, em centímetros. Sabe-se que o ponto V, na figura, representa o vértice da parábola, localizado sobre o eixo x. 

    \vfill
    \textbf{(continua no próximo slide)}
    
\end{frame}

\begin{frame}{ENEM 2013 (continuação)}
    Nessas condições, a altura do líquido contido na taça, em centímetros, é:
    \begin{columns}
        \begin{column}{0.4\textwidth}
            \begin{enumerate}[a]
                \item $1$ 
                \item $2$  
                \item $4$
                \item $5$ 
                \item \gab{$6$} %
            \end{enumerate}
        \end{column}

        \begin{column}{0.6\textwidth}
            \centering
            \includegraphics[width=0.7\linewidth]{imagens/enem 2013.png}
        \end{column}
    \end{columns}
    
\end{frame}

\begin{frame}{ENEM 2014}

    Um professor, depois de corrigir as provas de sua turma, percebeu que várias questões estavam muito difíceis. Para compensar, decidiu utilizar uma função polinomial f, de grau menor que 3, para alterar as notas x da prova para notas $y = f(x)$, da seguinte maneira:

    \begin{enumerate}[I]
        \item A nota zero permanece zero.
        \item A nota 10 permanece 10.
        \item A nota 5 passa a ser 6.
    \end{enumerate}

    \vfill
    \textbf{(continua no próximo slide)}
    
\end{frame}

\begin{frame}{ENEM 2014 (continuação)}
    A expressão da função $y = f(x)$ a ser utilizada pelo professor é:

    \begin{enumerate}[a]
        \item \gab{$y=-\dfrac{1}{25}x^{2}+\dfrac{7}{5}x$} % \\
        \item $y=-\dfrac{1}{10}x^{2}+2x$ \\
        \item $y=\dfrac{1}{24}x^{2}+\dfrac{7}{12}x$ \\
        \item $y=\dfrac{4}{5}x+2$ \\
        \item $y=x$ 
    \end{enumerate}
    
\end{frame}

\begin{frame}{ENEM PPL 2019}
    No desenvolvimento de um novo remédio, pesquisadores monitoram a quantidade Q de uma substância circulando na corrente sanguínea de um paciente, ao longo do tempo t. Esses pesquisadores controlam o processo, observando que Q é uma função quadrática de t. Os dados coletados nas duas primeiras horas foram:

    \begin{center}
        \includegraphics[scale=0.5]{imagens/enem-ppl-2019.png}
    \end{center} \textbf{(continua no próximo slide)}
\end{frame}

\begin{frame}{ENEM PPL 2019 (continuação)}
    Para decidir se devem interromper o processo, evitando riscos ao paciente, os pesquisadores querem saber, antecipadamente, a quantidade da substância que estará circulando na corrente sanguínea desse paciente após uma hora do último dado coletado. Nas condições expostas, essa quantidade (em miligrama) será igual a:

    \begin{enumerate}[a]
        \item $4$
        \item \gab{$7$} %
        \item $8$ 
        \item $9$ 
        \item $10$ 
    \end{enumerate}
\end{frame}

\begin{frame}{ENEM 2022}

    Ao analisar os dados de uma epidemia em uma cidade, peritos obtiveram um modelo que avalia a quantidade de pessoas infectadas a cada mês, ao longo de um ano. O modelo é dado por $p(t)=-t^{2}+10t+24$, sendo t um número natural, variando de 1 a 12, que representa os meses do ano, e p(t) a quantidade de pessoas infectadas no mês t do ano. Para tentar diminuir o número de infectados no próximo ano, a Secretaria Municipal de Saúde decidiu intensificar a propaganda oficial sobre os cuidados com a epidemia. Foram apresentadas cinco propostas (I, II, III, IV e V), com diferentes períodos de intensificação das propagandas: $\textbf{(I)}\ 1 \leq t \leq 2, \textbf{(II)}\ 3 \leq t \leq 4, \textbf{(III)}\ 5 \leq t \leq 6, \textbf{(IV)}\ 7 \leq t \leq 9$ e $\textbf{(V)}\ 10 \leq t \leq 12$

    \vfill
    \textbf{(continua no próximo slide)}
\end{frame}
\begin{frame}{ENEM 2022 (continuação)}
    A sugestão dos peritos é que seja escolhida a proposta cujo período de intensificação da propaganda englobe o mês em que, segundo o modelo, há a maior quantidade de infectados. A sugestão foi aceita. A proposta escolhida foi a:

    \begin{enumerate}[a]
            \item I
            \item II
            \item \gab{III} %
            \item IV 
            \item V
        \end{enumerate}
\end{frame}

\begin{frame}{ENEM 2013}
	O dono de um sítio pretende colocar uma haste de sustentação para melhor firmar dois postes de comprimentos iguais a 6m e 4m. A figura representa a situação real na qual os postes são descritos pelos segmentos AC e BD e a haste é representada pelo EF, todos perpendiculares ao solo, que é indicado pelo segmento de reta AB. Os segmentos AD e BC representam cabos de aço que
	serão instalados. 
	
	\begin{center}
		\includegraphics[scale=0.39]{imagens/enem_2013_sitio.png}
	\end{center} 
\end{frame}

\begin{frame}{ENEM 2013 (continuação)}
	Qual deve ser o valor do comprimento da haste EF?
	\begin{center}
		\includegraphics[scale=0.35]{imagens/enem_2013_sitio.png}
	\end{center} 
	
	\begin{itemize}
		\item [a)] 1 m
		\item [b)] 2 m
		\item [c)] 2,4 m
		\item [d)] 3 m 
		\item [e)] $2\sqrt{6}$ m
	\end{itemize}
\end{frame}

\end{document}

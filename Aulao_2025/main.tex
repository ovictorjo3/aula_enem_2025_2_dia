\documentclass[11pt]{beamer}
\usetheme{CambridgeUS}
\usecolortheme{dolphin}

\usepackage[utf8]{inputenc}
\usepackage[brazil]{babel}
\usepackage[T1]{fontenc}
\usepackage{xcolor}
\usepackage{amsmath}
\usepackage{amsfonts}
\usepackage{amssymb}
\usepackage{graphicx}
\usepackage{setspace}
\usepackage{ragged2e}

\setbeamertemplate{caption}[numbered]
\setbeamertemplate{navigation symbols}{}

% Controle do gabarito %
%%%%%%%%%%%%%%%%%%%%%
\newif\ifgab
\gabtrue
\newcommand{\gab}[1]{%
  \ifgab
    \textcolor{red!80!black}{\textbf{#1}}%
  \else
    #1%
  \fi
}
%%%%%%%%%%%%%%%%%%%%%
\author[Luiz Claudio \& Joao Victor \& Letícia]{% 
	Luiz Claudio\inst{1} Joao Victor\inst{2} \and Letícia Almeida\inst{2}}
\title[AULÃO - MATEMÁTICA]{AULÃO MATEMÁTICA - ENEM 2025 - 2º Dia}
\institute[]{% 
	\textsuperscript{1}  Professor, CETi Gilberto Mestrinho (CETi) \\
	\textsuperscript{2} Pibidianos, Instituto Federal do Amazonas (IFAM)
}
\date{\today} 
\titlegraphic{\includegraphics[width=0.5\textwidth]{imagens/novo_logo.png}}
%\subject{}

% ---------------------------------------------------------

\begin{document}
\justifying
\onehalfspacing 

\begin{frame}
    \titlepage
\end{frame}

\section{Questões - Letícia}

	\begin{frame}{ENEM 2013}
		Em um certo teatro, as poltronas são divididas em setores. A figura apresenta a vista do setor 3 desse teatro, no qual as cadeiras escuras estão reservadas e as claras não foram vendidas. \\

		\centering
		\includegraphics[width=0.6\textwidth]{imagens/cadeiras.jpeg}

	\end{frame}
	
	\begin{frame}{ENEM 2013 - Continuação}
		
		A razão que representa a quantidade de cadeiras reservadas do setor 3 em relação ao total de cadeiras desse mesmo setor é:
		
		\begin{itemize}
			\item [a)] {17}/{70}
			\item [b)] {17}/{53}
			\item [c)] {53}/{70}
			\item [d)] {53}/{17}
			\item [e)] {70}/{17}
		\end{itemize}
	\end{frame}

	\begin{frame}{ENEM 2010}
	
		Uma empresa vende tanques de combustíveis de formato cilíndrico, em três tamanhos, com medidas indicadas nas figuras. O preço do tanque é diretamente proporcional à medida da área da superfície lateral do tanque. O dono de um posto de combustível deseja encomendar um tanque com menor custo por metro cúbico de capacidade de armazenamento. \\
		
		\vspace{2mm}
	
		\centering
		\includegraphics[width=0.8\textwidth]{imagens/tanque.jpeg}
	
\end{frame}

\begin{frame}{ENEM 2010 - Continuação}
	
	Qual dos tanques deverá ser escolhido pelo dono do posto? (Considere $\pi = 3$)
	
	\begin{itemize}
		\item [a)] I, pela relação área/capacidade de armazenamento de ${1}/{3}$
		\item [b)] I, pela relação área/capacidade de armazenamento de ${4}/{3}$.
		\item [c)] II, pela relação área/capacidade de armazenamento de ${3}/{4}$.
		\item [d)] III, pela relação área/capacidade de armazenamento de ${2}/{3}$.
		\item [e)] III, pela relação área/capacidade de armazenamento de ${7}/{12}$.
	\end{itemize}
\end{frame}


\begin{frame}{ENEM 2010}
	
	A loja Telas \& Molduras cobra 20 reais por metro quadrado de tela, 15 reais por metro linear de moldura, mais uma taxa fixa de entrega de 10 reais. Uma artista plástica precisa encomendar telas e molduras a essa loja, suficientes para 8 quadros retangulares (25 cm x 50 cm). Em seguida, fez uma segunda encomenda, mas agora para 8 quadros retangulares (50 cm x 100 cm). 
	
\end{frame}

\begin{frame}{ENEM 2010 - Continuação}
	
	O valor da segunda encomenda será:
	
	\begin{itemize}
		\item [a)] o dobro do valor da primeira encomenda, porque a altura e a largura dos quadros dobraram.
		\item [b)] maior do que o valor da primeira encomenda, mas não o dobro.
		\item [c)] a metade do valor da primeira encomenda, porque a altura e a largura dos quadros dobraram.
		\item [d)] menor do que o valor da primeira encomenda, mas não a metade.
		\item [e)] igual ao valor da primeira encomenda, porque o custo de entrega será o mesmo. 
	\end{itemize}
\end{frame}

\begin{frame}{ENEM 2011	}
	
	\begin{columns}[T]
		\begin{column}{0.48\textwidth}
			\justifying
			Para uma atividade realizada no laboratório de Matemática, um aluno precisa construir uma maquete da quadra de esportes da escola que tem 28 m de comprimento por 12 m de largura. A maquete deverá ser construída na escala de 1 : 250. Que medidas de comprimento e largura, em cm, o aluno utilizará na construção da maquete? 
		\end{column}
		
		\begin{column}{0.48\textwidth}
			
			\begin{itemize}
				\item [a)] $4,8$ e $11,2$.
				\item [b)] $7,0$ e $3,0$.
				\item [c)] $11,2$ e $4,8$.
				\item [d)] $28,0$ e $12,0$.
				\item [e)] $30,0$ e $70,0$. 
			\end{itemize}
			
		\end{column}
	\end{columns}
\end{frame}

\begin{frame}{ENEM 2010}
	
	\begin{columns}[T] 
		\begin{column}{0.48\textwidth}
			\justifying
			Em 2006, a produção mundial de etanol foi de 40 bilhões de litros e a de biodiesel, de 6,5 bilhões. Neste mesmo ano, a produção brasileira de etanol correspondeu a 43\% da produção mundial, ao passo que a produção dos Estados Unidos da América, usando milho, foi de 45\%. \\
			
			\vspace{1mm}
			
			Considerando que, em 2009, a produção mundial de etanol seja a mesma de 2006 e que os Estados Unidos produzirão somente a metade de sua produção de 2006, para que o total produzido pelo  
		\end{column}
		
		\begin{column}{0.48\textwidth}
			\justifying
			Brasil e pelos Estados Unidos continue correspondendo a 88\% da produção mundial, o Brasil deve aumentar sua produção em, aproximadamente,
			
			\begin{itemize}
				\item [a)] 22,5\%.
				\item [b)] 50,0\%.
				\item [c)] 52,3\%.
				\item [d)] 65,5\%.
				\item [e)] 77,5\%
			\end{itemize}
		\end{column}
	\end{columns}
\end{frame}

\begin{frame}{ENEM 2011}
	\begin{columns}[T] 
		\begin{column}{0.48\textwidth}
			\justifying
			Uma pessoa aplicou certa quantia em ações. No primeiro mês, ela perdeu 30\% do total do investimento e, no segundo mês, recuperou 20\% do que havia perdido. \\
			
			\vspace{1mm}
			
			Depois desses dois meses, resolveu tirar o montante de R\$ 3800,00 gerado pela aplicação. 
		\end{column}
		
		\begin{column}{0.48\textwidth}
			
			\justifying
			A quantia inicial que essa pessoa aplicou em ações corresponde ao valor de:
			
			\begin{itemize}
				\item [a)] R\$ 4.222,22.
				\item [b)] R\$ 4.523,80.
				\item [c)] R\$ 5.000,00.
				\item [d)] R\$ 13.300,00.
				\item [e)] R\$ 17.100,00.
			\end{itemize}
			
		\end{column}
	\end{columns}
\end{frame}

\begin{frame}{ENEM 2016}
	\justifying
	Uma pessoa comercializa picolés. No segundo dia de certo evento ela comprou 4 caixas de picolés, pagando R\$ 16,00 a caixa com 20 picolés para revendê-los no evento. No dia anterior, ela havia comprado a mesma quantidade de picolés, pagando a mesma quantia, e obtendo um lucro de R\$ 40,00 (obtido exclusivamente pela diferença entre o valor de venda e o de compra dos picolés) com a venda de todos os picolés que possuía.\\
	
	\vspace{2mm}
	
	Pesquisando o perfil do público que estará presente no evento, a pessoa avalia que será possível obter um lucro 20\% maior do que o obtido com a venda no primeiro dia do evento.
\end{frame}

\begin{frame}{ENEM 2016 - Continuação}
	Para atingir seu objetivo, e supondo que todos os picolés disponíveis foram vendidos no segundo dia, o valor de venda de cada picolé, no segundo dia, deve ser:
	
	\begin{itemize}
		\item [a)] R\$ 0,96.
		\item [b)] R\$ 1,00.
		\item [c)] R\$ 1,40.
		\item [d)] R\$ 1,50.
		\item [e)] R\$ 1,56.
	\end{itemize}
\end{frame}

\begin{frame}{ENEM 2014}
	
	\begin{columns}[T] 
		\begin{column}{0.48\textwidth}
			\justifying
			Um show especial de Natal teve 45.000 ingressos vendidos. Esse evento ocorrerá em um estádio de futebol que disponibilizará 5 portões de entrada, com 4 catracas eletrônicas por portão. Em cada uma dessas catracas, passará uma única pessoa a cada 2 segundos. O público foi igualmente dividido pela quantidade de portões e catracas, indicados no ingresso para o show, para a efetiva entrada no estádio. 
			
		\end{column}
		
		\begin{column}{0.48\textwidth}
			\justifying
			Suponha que todos aqueles que compraram ingressos irão ao show e que todos passarão pelos portões e catracas eletrônicas indicados. Qual é o tempo mínimo para que todos passem pelas catracas?
			
			\begin{itemize}
				\item [a)] 1 hora.
				\item [b)] 1 hora e 15 minutos.
				\item [c)] 5 horas.
				\item [d)] 6 horas.
				\item [e)] 6 horas e 15 minutos.
			\end{itemize}
		
		\end{column}
	\end{columns}
\end{frame}

 \begin{frame}{ENEM 2015}
 	
 	\justifying
	A insulina é utilizada no tratamento de pacientes com diabetes para o controle glicêmico. Para facilitar sua aplicação, foi desenvolvida uma “caneta” na qual pode ser inserido um refil contendo 3mL de insulina, como mostra a imagem.
	
	\centering
	\includegraphics[width=0.5\linewidth]{imagens/insulina.jpeg}
	
\end{frame}

\begin{frame}{ENEM 2015 - Continuação}
	
	\begin{columns}[T] 
		\begin{column}{0.48\textwidth}
			\justifying
			Para controle das aplicações, definiu-se a unidade de insulina como 0,01mL. Antes de cada aplicação, é necessário descartar 2 unidades de insulina, de forma a retirar possíveis bolhas de ar. \\
			
			A um paciente foram prescritas duas aplicações diárias: 10 unidades de insulina pela manhã e 10 à noite.
		\end{column}
		
		\begin{column}{0.48\textwidth}
			\justifying
			Qual o número máximo de aplicações por refil que o paciente poderá utilizar com a dosagem prescrita? 
			
			\begin{itemize}
				\item [a)] 25.
				\item [b)] 15.
				\item [c)] 13.
				\item [d)] 12.
				\item [e)] 8. 
			\end{itemize}
			
		\end{column}
	\end{columns}
\end{frame}

\begin{frame}{ENEM 2016}
	
	No tanque de um certo carro de passeio cabem até 50 L de combustível, e o rendimento médio deste carro na estrada é de 15 km L de combustível. Ao sair para uma viagem de 600 km o motorista observou que o marcador de combustível estava exatamente sobre uma das marcas da escala divisória do medidor, conforme figura a seguir. \\
	
	\vspace{2mm}
	
	\centering
	\includegraphics[width=0.4\linewidth]{imagens/carrinho.jpeg}
\end{frame}

\begin{frame}{ENEM 2016 - Continuação}
	
	\begin{columns}[T]
		\begin{column}{0.48\textwidth}
			\justifying
			Como o motorista conhece o percurso, sabe que existem, até a chegada a seu destino, cinco postos de abastecimento de combustível, localizados a 150 km,187 km, 450 km, 500 km e 570 km do ponto de partida. \\
			
			Qual a máxima distância, em quilômetro, que poderá percorrer até ser necessário reabastecer o veículo, de modo a não ficar sem combustível na estrada?
		\end{column}
		\begin{column}{0.48\textwidth}
			
			\begin{itemize}
				\item [a)] 570.
				\item [b)] 500.
				\item [c)] 450.
				\item [d)] 187.
				\item [e)] 150.
			\end{itemize}
			
		\end{column}
	\end{columns}
\end{frame}


\section{Questões - Joao Victor}

\begin{frame}{Tópicos}
	\begin{alertblock}{}
		\vspace{1mm}
		\begin{itemize}
			\item Função polinomial do 1º grau; \\
			\item \uncover<2->{Função polinomial do 2º grau;} \\
			\item \uncover<3->{Geometria plana;} \\
			\item \uncover<4->{Estatística.}
		\end{itemize}
	\end{alertblock}
\end{frame}

\section{Função afim}

\begin{frame}{ENEM 2024 - Questão 136 (Caderno 5 - Amarelo)}
	%função afim
    Uma empresa produz mochilas escolares sob encomenda. Essa empresa tem um custo total de produção, composto por um custo fixo, que não depende do númerode mochilas, mais um custo variável, que é proporcional ao número de mochilas produzidas. O custo total cresce de forma linear, e a tabela apresenta esse custo para três quantidades de mochilas produzidas.
    
    \begin{center}
    	\includegraphics[scale=0.5]{imagens/enem_mochila.png}
    \end{center} 
\end{frame}

\begin{frame}{ENEM 2024 - Questão 136 (Caderno 5 - Amarelo)}
	
	O custo total, em real, para a produção de 80 mochilas será:
	
	\begin{columns}
		\begin{column}{0.6\textwidth}
			\centering
			\includegraphics[scale=0.55]{imagens/enem_mochila.png}
		\end{column}
		
		\begin{column}{0.2\textwidth}
			
			\begin{itemize}
				\item [a)] $2 400,00$
				\item [b)] $2 520,00$
				\item [c)] {$2 550,00$} %
				\item [d)] $2 700,00$
				\item [e)] $2 800,00$
			\end{itemize}
		\end{column}
	\end{columns}
	
	
\end{frame}

\begin{frame}{ENEM 2024 - Questão 152 (Caderno 5 - Amarelo)}
	%função afim
	As receitas anuais obtidas por uma indústria no períodode 2014 a 2021, em milhão de reais, foram registradas, por pontos, em um gráfico. Nele, também está representada a reta que descreve a tendência de evolução das receitas. Essa reta pode ser utilizada para estimar as receitas dos anos seguintes. 
	
	\begin{center}
		\includegraphics[scale=0.38]{imagens/enem2024-receitas.png}
	\end{center} 
\end{frame}

\begin{frame}{ENEM 2024 - Questão 152 (Caderno 5 - Amarelo)}
	
	A estimativa da receita, em milhão de reais, dessaindústria, para o ano de 2026, obtida a partir dessa reta de tendência, é:
	
	\begin{columns}
		\begin{column}{0.6\textwidth}
			\centering
			\includegraphics[scale=0.45]{imagens/enem2024-receitas.png}
		\end{column}
		
		\begin{column}{0.2\textwidth}
			
			\begin{itemize}
				\item [a)] $7$
				\item [b)] $8$ %
				\item [c)] $9$
				\item [d)] $10$
				\item [e)] {$11$} 
			\end{itemize}
		\end{column}
	\end{columns}
	
\end{frame}

\begin{frame}{ENEM 2024 - Questão 156 (Caderno 5 - Amarelo)}
	%função afim
	O uso de aplicativos de transporte tem sido uma alternativa à população que busca preços mais competitivos para se locomover, principalmente nas grandes cidades. As formas usadas para determinar o valor cobrado por cada viagem variam de um aplicativo para outro, mas, em geral, o valor V a ser pago, em real, varia em função de: 
	
	\begin{itemize}
		\item tarifa base F: valor fixo, em real, cobrado no inícioda viagem;
		\item tempo T: tempo, em minuto, de duração da viagem;
		\item distância D: distância percorrida, em quilômetro.
	\end{itemize}
	
	Um desses aplicativos cobra R\$ 2,00 de valor fixo,acrescido de R\$ 0,26 por minuto de viagem e de R\$ 1,40 por quilômetro rodado.
\end{frame}

\begin{frame}{ENEM 2024 - Questão 156 (Caderno 5 - Amarelo)}
	%função afim
	Nessas condições, a expressão que fornece o valor V a ser pago por uma viagem desse aplicativo é:
	
	\begin{itemize}
		\item [a)] 2,00F + 0,26T + 1,40D
		\item [b)] 2,00 + 0,26T + 1,40D   %
		\item [c)] 2,00 + 0,26T + D
		\item [d)] 0,26T + 1,40D
		\item [e)] F + T + D
	\end{itemize}
	
\end{frame}

\section{Função Quadrática}

\begin{frame}{ENEM 2013}
	
	A parte interior de uma taça foi gerada pela rotação de uma parábola em torno de um eixo z, conforme mostra a figura. A função real que expressa a parábola, no plano cartesiano da figura, é dada pela lei: $f(x)=\dfrac{3}{2}x^{2}-6x+c$ onde C é a medida da altura do líquido contido na taça, em centímetros. Sabe-se que o ponto V, na figura, representa o vértice da parábola, localizado sobre o eixo x. 
\end{frame}

\begin{frame}{ENEM 2013 (continuação)}
	Nessas condições, a altura do líquido contido na taça, em centímetros, é:
	\begin{columns}
		\begin{column}{0.6\textwidth}
			\centering
			\includegraphics[width=0.7\linewidth]{imagens/enem 2013.png}
			
		\end{column}
	
		\begin{column}{0.4\textwidth}
			\begin{itemize}
				\item [a)] $1$ 
				\item [b)] $2$  
				\item [c)] $4$
				\item [d)] $5$ 
				\item [e)] {$6$} %
			\end{itemize}
		\end{column}
	\end{columns}
\end{frame}

\begin{frame}{ENEM 2014}
	
	Um professor, depois de corrigir as provas de sua turma, percebeu que várias questões estavam muito difíceis. Para compensar, decidiu utilizar uma função polinomial f, de grau menor que 3, para alterar as notas x da prova para notas $y = f(x)$, da seguinte maneira:
	
	\begin{enumerate}[I]
		\item A nota zero permanece zero.
		\item A nota 10 permanece 10.
		\item A nota 5 passa a ser 6.
	\end{enumerate}
	
\end{frame}

\begin{frame}{ENEM 2014 (continuação)}
	A expressão da função $y = f(x)$ a ser utilizada pelo professor é:
	
	\begin{itemize}
		\item [a)] {$y=-\dfrac{1}{25}x^{2}+\dfrac{7}{5}x$} % \\
		\item [b)] $y=-\dfrac{1}{10}x^{2}+2x$ \\
		\item [c)] $y=\dfrac{1}{24}x^{2}+\dfrac{7}{12}x$ \\
		\item [d)] $y=\dfrac{4}{5}x+2$ \\
		\item [e)] $y=x$ 
	\end{itemize}
	
\end{frame}

\section{Estatistica}

\begin{frame}{ENEM 2024 - Questão 164 (Caderno 5 - Amarelo)}
	%media 
	Ao calcular a média de suas notas em 4 provas, um estudante dividiu, por engano, a soma das notas por 5. Com isso, a média obtida foi 1 unidade menor do que deveria ser, caso fosse calculada corretamente. O valor correto da média das notas desse estudante é:
	
	\begin{itemize}
		\item [a)] 4
		\item [b)] 5   %
		\item [c)] 6
		\item [d)] 19
		\item [e)] 21
	\end{itemize}
	
\end{frame}

\begin{frame}{ENEM 2024 - Questão 137 (Caderno 5 - Amarelo)}
	%mediana
	A umidade relativa do ar é um dos indicadores utilizados na meteorologia para fazer previsões sobre o clima. O quadro apresenta as médias mensais, em porcentagem, da umidade relativa do ar em um período de seis meses
	consecutivos em uma cidade.
	
	\begin{center}
		\includegraphics[scale=0.5]{imagens/enem2024-umidade.png}
	\end{center} 
\end{frame}

\begin{frame}{ENEM 2024 - Questão 137 (Caderno 5 - Amarelo)}

	Nessa cidade, a mediana desses dados, em porcentagem,
	da umidade relativa do ar no período considerado foi:
	
	\begin{columns}
		\begin{column}{0.6\textwidth}
			\centering
			\includegraphics[scale=0.5]{imagens/enem2024-umidade.png}
		\end{column}
		
		\begin{column}{0.2\textwidth}
			
			\begin{itemize}
				\item [a)] $56$
				\item [b)] $58$
				\item [c)] $59$
				\item [d)] $60$
				\item [e)] {$62$} %
			\end{itemize}
		\end{column}
	\end{columns}
\end{frame}

\begin{frame}{ENEM 2024 - Questão 154 (Caderno 5 - Amarelo)}
	%media aritmetica
	Contratos de vários serviços disponíveis na internet apresentam uma quantidade excessiva de informações. Isso faz com que o tempo necessário para a leitura desses contratos possa ser longo. O quadro apresenta uma amostra do tempo considerado necessário para a leitura completa do contrato de alguns
	serviços digitais.
	
	\begin{center}
		\includegraphics[scale=0.5]{imagens/enem2024-contratos.png}
	\end{center} 
\end{frame}

\begin{frame}{ENEM 2024 - Questão 154 (Caderno 5 - Amarelo)}
	
	O tempo médio, em minuto, necessário para a leitura completa de um contrato de serviço dentre os listados no quadro é, com uma casa decimal, aproximadamente,
	
	\begin{columns}
		\begin{column}{0.6\textwidth}
			\centering
			\includegraphics[scale=0.5]{imagens/enem2024-contratos.png}
		\end{column}
		
		\begin{column}{0.2\textwidth}
			
			\begin{itemize}
				\item [a)] $13,0$
				\item [b)] $15,0$
				\item [c)] {$19,8$} %
				\item [d)] $20,0$ 
				\item [e)] $23,3$ 
			\end{itemize}
		\end{column}
	\end{columns}
\end{frame}


\section{Geometria Plana}

\begin{frame}{ENEM 2024 - Questão 172 (Caderno 5 - Amarelo)}
	%Área de quadriláteros
	O estádio do Maracanã passou por algumas modificações estruturais para realização da Copa doMundo de 2014, como, por exemplo, as dimensões do campo retangular. Para se adaptar aos padrões da Fifa, as dimensões do campo foram reduzidas de 110 m × 75 m para 105 m × 68 m. Em quantos metros quadrados a área do campo doMaracanã foi reduzida?
	
	\begin{itemize}
		\item [a)] 24
		\item [b)] 35
		\item [c)] 555
		\item [d)] 1110 %
		\item [e)] 1145
	\end{itemize}
	
\end{frame}

\begin{frame}{ENEM 2013}
	O dono de um sítio pretende colocar uma haste de sustentação para melhor firmar dois postes de comprimentos iguais a 6m e 4m. A figura representa a situação real na qual os postes são descritos pelos segmentos AC e BD e a haste é representada pelo EF, todos perpendiculares ao solo, que é indicado pelo segmento de reta AB. Os segmentos AD e BC representam cabos de aço que
	serão instalados. 
	
	\begin{center}
		\includegraphics[scale=0.39]{imagens/enem_2013_sitio.png}
	\end{center} 
\end{frame}

\begin{frame}{ENEM 2013 (continuação)}
	Qual deve ser o valor do comprimento da haste EF?
	
	\begin{columns}
		\begin{column}{0.6\textwidth}
			\centering
			\includegraphics[scale=0.35]{imagens/enem_2013_sitio.png}
		\end{column}
		
		\begin{column}{0.2\textwidth}
			
			\begin{itemize}
				\item [a)] 1 m
				\item [b)] 2 m
				\item [c)] 2,4 m %
				\item [d)] 3 m 
				\item [e)] $2\sqrt{6}$ m
			\end{itemize}
		\end{column}
	\end{columns}

\end{frame}

\end{document}
